\documentclass[a4paper,12pt,oneside]{article}
 \usepackage[latin1]{inputenc}
 \usepackage[T1]{fontenc}
 \usepackage[english]{babel}
 \usepackage{amssymb}
 \usepackage{graphicx}
\usepackage{subfigure}
\usepackage{amsmath}
\usepackage{amsthm}

\begin{document}
 
\begin{titlepage}
\begin{center}
\vspace{10cm}
\hspace{0.5cm}
\newline
\Huge{\textbf{Betting on Catalan numbers}}
\vspace{1cm}
\newline 
\Large{\hspace{-2cm} Dylan Da Silva Moreira \hspace{3cm} Hukic Ibrahim}
\vspace{1cm}
\newline
\hspace{4cm} 10 Novembre 2019
\vspace{\fill}

\hspace{3cm}
\newline
University of Luxembourg
\vspace{0.5cm}
\newline
Academic year 2019-2020 (Winter Semestre)
\end{center}
\end{titlepage}
\newpage
\section{\underline{Introduction to the task}}
We got the following task at the beginning of the project:

\vspace{1cm}
A fair coin is flipped 100 times yielding a sequence $a=(a_{1},...,a_{100})$ of Heads or Tails. Player $A$ realizes that if one thinks of Heads as $+1$ and Tails as $-1$, then for each $k$ $\in$ $[1,100]$,
\begin{center}
$\sum_{i=1}^{k}a_{i}\geq0$
\end{center}
Furthermore $\sum_{100}^{i=1}a_{i}=0$. Player $B$ thinks such a sequence is extremely unlikely and offers a bet to $A$ with a $100:1$ odds that if they flip the coin again 100 times yielding a random sequence $x=(x_{1},...,x_{100})$, then $x$ is not going to satisfy the properties of $a$. Should A accept the bet? \\What are the true odds for such an event? Is there a good graphical representation for the properties of $a$?\\What if there are $2n$ coin flips? Biased coin?
\vspace{0.5cm}
\\
To get the solution of this task, we must get closer to the expression of the catalan numbers, how we did this will be discuss in the following.
\vspace{0.5cm}
\\
After this we wrote two informatical programms to see how many times we need to compute such a sequence to come close as possible to the probability of $\sum_{i=1}^{k}a_{i}\geq0$ given from thr Catalan numbers formula, this could be seen as the experimental goal.
\\
(Every reader is welcome to copie the programm and to try it by himself)
\\
We worked out the case, when the probability between -1 and +1 is not equal.
\newpage
Comming soon:- logistical tools/Mathematical tools -Catalan numbers with the proof - Informatical approche to the catalan numbers probability (evtl. 2programms) -Result of the first question -Biased coins 
\section{\underline{Logistical tools}}
We used the following programs:
\vspace{0,5cm}
\begin{itemize}
  \item Mathsage
	\item Latex
\end{itemize}
\vspace{2cm}
\section{\underline{Mathematical tools}}
We used the following mathematical approach
\vspace{0,5cm} 
\begin{itemize}  
  \item Numeration
  \item Fundamental Probability 
  \item Catalan Numbers
 \end{itemize}
\newpage
\section{\underline{The theoretically approach of the Catalan numbers}}
\vspace{0.5cm}
We will prove that we are allowed to use the Catalan numbers formula for our task.
\begin{figure}[h]
\centering
\includegraphics[scale=0.3]{Graphik}
\caption{Represenation}
\end{figure}
\\
We start with a representation of our task with a redefinition of our $\{-1,1\}$ randome sequence into right and up mouvement, where $-1$ represents an up step an $+1$ a right step. The solution to our coin flip problem are all the paths that end on the line $f(x)=x$.
\\
Hence on the point $(n,n)$  (sum equals zero) and do not cross the line ,given by the function f(x)=x, at any given moment (sum is always positive).\\
The number of possibility satisfying this conditions are equal to : \\
\begin{center}
 $\binom{2n}{n}$-$\binom{2n}{n+1}$
\end{center}
The proof will be given as next step.
\newpage

\begin{proof}
At the first step, we will show that the number of paths, who satisfy to end at the point $(n,n)$, is equal to $\binom{2n}{n}$, which means that $\binom{2n}{n}$ gives us all the paths even this which we do not want.
\\
This we will prove by induction.
\begin{proof}
We have to show that $\binom{2n}{n}$ gives us all the paths to reach the point $(n,n)$.\vspace{0.3cm}\\
\underline{Initialization:}
\\ For n:=1, we have exactly 2 paths to get from (0,0)  to (1,1).
\\ More precise, 
\begin{center}
the frist path could be: one step right and the next one one step up, 
\\the second path is than: one tep up and one right.
\end{center}
\vspace{0.3cm}
Verify for  $\binom{2n}{n}$, so we have  $\binom{2*1}{1}=\frac{2!}{(2-1)!\,1!}=\frac{2}{1}=2$\
\vspace{0.3cm} \\ So the assertion is true for n=1
\vspace{0.2cm}\\
\underline{Now by induction:}
\vspace {0.1cm}\\ We will now suppose that the assertion is true for some $n\in\mathbb{N^*}$ and we will prove that it holds for any $n+1$ \\If we get $\binom{2(n+1)}{n+1}$ as a linear combination of $\binom{2n}{n}$, we have prove that the assertion is true for $n+1$.\vspace{0.3cm}\\
Now let us compute:\vspace{0.2cm}\\
\begin{center}
 $\binom{2(n+1)}{n+1}=\frac{(2(n+1)!}{(2(n+1)-(n+1))!\,(n+1)!}
=\frac{(2(n+1))!}{(n+1)!(n+1)!}=\frac{2n!}{n!n!}\frac{(2n+1)(2n+2)}{(n+1)(n+1)}$\\$=\frac{2n!}{n!n!}\frac{2n+2+4n^2+4n}{n^2+2n+1}=\frac{2n!}{n!n!}\frac{2(2n^2+3n+1)}{n^2+2n+1}=\frac{2n!}{n!n!}\frac{2(n+1)(n+\frac{1}{2})}{(n+1)^2}=\frac{2n!}{n!n!}\frac{2(n+\frac{1}{2})}{n+1}=\frac{2n!}{n!n!}\frac{2n+1}{n+1}=\binom{2n}{n}(\frac{n}{n+1}+\frac{n+1}{n+1})=\binom{2n}{n}(\frac{n}{n+1}+1)=\frac{n}{n+1}\binom{2n}{n}+\binom{2n}{n}$\
\end{center}
\vspace{0.3cm}
Hence,$\binom{2(n+1)}{n+1}$ is a linear combination of $\binom{2n}{n}$,so we get what we wanted and we have proven that that the assertion holds for every $n\in\mathbb{N^*}$.
\end{proof}
Now one must only substract the number of  '' bad paths '', paths which cross the line.
\\
To find out the number of bad paths we will flip the position at the path after it has crossed the line for the first time.
\\
By fliping one means interchanging all up and right steps.
\newpage
By observation, we find out that every bad path like that will end at the point ($n-1$,$n+1$), this follows because, we interchange all up and right steps after the path has crossed the line, by doing this we underline the fact that the path crossed the line ones and do not follows the rule we defined. By this procedure, we get that there will be one up step more and by the reason that we still have $2n$ steps, which means that we in end effect have $n-1$ right steps and $n+1$ up steps. This explains why the bad paths end at the point ($n-1,n+1$) using this methode.
\vspace{0,3cm}
\\Since the flipping process is reversible, there is a bijective relation between paths ending on ($n,n$) and ($n-1,n+1$).
\vspace{0,3cm}
\\Hence the number of bad paths are given by the permutation of our ($n+1$) up steps for a total of $2n(=n+1+n-1)$ steps.
\vspace{0,3cm}
\\The number op bad paths is given by:
\begin{center}
$\binom{n+1+n-1}{n+1}$=$\binom{2n}{n+1}$=$\binom{2n}{n-1}$
\end{center}
By substracting this number from all the possible paths we get:
\begin{center}
$\binom{2n}{n}-\binom{2n}{n+1}$=$\binom{2n}{n}-\binom{2n}{n-1}$ 
\end{center}
Now we will compute our relation to get a more simple result:
\begin{center}
$\binom{2n}{n}-\binom{2n}{n-1}$=$\frac{2n!}{n!n!}-\frac{2n!}{(n+1)!(n-1)!}$=$2n!(\frac{1}{n!n!}-\frac{1}{(n+1)!(n-1)!})$=$2n!(\frac{1}{n(n-1)!n!}-\frac{1}{(n+1)n!(n-1)!})$=$2n!(\frac{n+1}{(n+1)n(n-1)!n!}-\frac{n}{(n+1)n(n-1)!n!})$=$\frac{2n!}{(n+1)n(n-1)!n!}$=$\binom{2n}{n}\frac{1}{n+1}$
\end{center}
\vspace{0.2cm}
This finishs our proof
\end{proof}

\end{document}


